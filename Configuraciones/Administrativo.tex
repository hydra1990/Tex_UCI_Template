% Aquí es donde se definen el autor, la universidad, el título y más

% Información personal:
\newcommand{\myAuthorName}  {Nombre del tesista}% Nombre del Autor
\newcommand{\myAuthorEmail} {myemail@some.cu} % Correo electrónico del Autor
\newcommand{\mySupervisors} {Tutor 1 \\ Tutor 2}% Nombre de(l) (los) Tutor(es)
\newcommand{\myTitle}       {Título de la tesis} % Thesis title goes here
\newcommand{\mySubject}     {Tema de la tesis} % Subject goes here
\newcommand{\myKeywords}    {clave, palabras} % Keywords goes here


% University information
\newcommand{\myUniversity}{Universidad de las Ciencias Informáticas (UCI)} %The Iniversity name goes here
\newcommand{\myUniversityWeb}{http://www.uci.cu} %University Web Site URL Here (include http://
\newcommand{\myDepartment}{Departamento del tesista} % The Department goes here 
\newcommand{\myDepartmentWeb}{http://www.uci.cu} % Department Web Site URL Here (include http://)
\newcommand{\myGroup}{Grupo de investigación}
\newcommand{\myGroupWeb}{http://www.uci.cu}
\newcommand{\myFaculty}{Facultad a la que pertenece el tesista}
\newcommand{\myFacultyWeb}{http://www.uci.cu}

%Degree, program or course: ex: Master of Science, Engineering Physics
\newcommand{\myDegree}{Ingeniero en Ciencias Informáticas} % The degree, program or course-name goes here


% can be left untouched, both:
\newcommand{\myDate}{\today}
\newcommand{\myPartyalFulfillment}{Tesis presentada en opción al grado Científico de }

% Nombres de los capitulos
\newcommand{\CapUno}{Título del Capítulo Uno}
\newcommand{\CapDos}{Título del Capítulo Dos}
\newcommand{\CapTres}{Título del Capítulo Tres}