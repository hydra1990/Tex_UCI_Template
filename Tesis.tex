%% ----------------------------------------------------------------
%% Tesis.tex -- Fichero Principal (el que se compila con Latex)
%% ---------------------------------------------------------------- 

%% Esta plantilla está basada en la Tesis de Grado escrita por Sunil Patel,
% (quién la basó en la plantilla ecsthesis) bajo la Licencia Pública del Proyecto LaTeX
% que puede encontrarse acá: http://latex-project.org/lppl/
% con la esperanza de que sea más fácil de usar y escalar a sus necesidades
% por Simon Ternsjö en 2013-10
% Traducida al español para su uso por hispano-hablantes por Dairelys García Rivas en 2017-07

% INSTRUCCIONES:

% La idea es no editar mucho este documento, sino llenar la información
% en los diferentes ficheros en las carpetas;
% Configuraciones, Páginas_Princ, Capítulos, Anexos y posiblemente Figuras,
% así como el fichero Bibliografía.bib

% Esta plantilla es fácil de escalar a sus necesidades, 
% simplemente comente las declaraciones input que se explican debajo

% Configurar el documento:
\documentclass[letterpaper, 12pt, oneside]{Tesis}  % Usar el estilo Tesis", basado en el estilo ECS Thesis por Steve Gunn

% Adicionar más paquetes en Paquetes.tex:
\input{Configuraciones/Paquetes.tex}

% Usar si quiere:
%5%\graphicspath{Figuras/}  % Ruta del fichero de gráficos (configurado para que los gráficos estén en formato PDF)
%5%\hypersetup{urlcolor=blue, colorlinks=true}  % Colorea los hipervínculos en azul, pero puede distraer si hay muchos vínculos.
\hypersetup{urlcolor=black, colorlinks=true}

% Establecer su nombre, el título del trabajo y más en Administrativo.tex:
% Aquí es donde se definen el autor, la universidad, el título y más

% Información personal:
\newcommand{\myAuthorName}  {Nombre del tesista}% Nombre del Autor
\newcommand{\myAuthorEmail} {myemail@some.cu} % Correo electrónico del Autor
\newcommand{\mySupervisors} {Tutor 1 \\ Tutor 2}% Nombre de(l) (los) Tutor(es)
\newcommand{\myTitle}       {Título de la tesis} % Thesis title goes here
\newcommand{\mySubject}     {Tema de la tesis} % Subject goes here
\newcommand{\myKeywords}    {clave, palabras} % Keywords goes here


% University information
\newcommand{\myUniversity}{Universidad de las Ciencias Informáticas (UCI)} %The Iniversity name goes here
\newcommand{\myUniversityWeb}{http://www.uci.cu} %University Web Site URL Here (include http://
\newcommand{\myDepartment}{Departamento del tesista} % The Department goes here 
\newcommand{\myDepartmentWeb}{http://www.uci.cu} % Department Web Site URL Here (include http://)
\newcommand{\myGroup}{Grupo de investigación}
\newcommand{\myGroupWeb}{http://www.uci.cu}
\newcommand{\myFaculty}{Facultad a la que pertenece el tesista}
\newcommand{\myFacultyWeb}{http://www.uci.cu}

%Degree, program or course: ex: Master of Science, Engineering Physics
\newcommand{\myDegree}{Ingeniero en Ciencias Informáticas} % The degree, program or course-name goes here


% can be left untouched, both:
\newcommand{\myDate}{\today}
\newcommand{\myPartyalFulfillment}{Tesis presentada en opción al grado Científico de }

% Nombres de los capitulos
\newcommand{\CapUno}{Título del Capítulo Uno}
\newcommand{\CapDos}{Título del Capítulo Dos}
\newcommand{\CapTres}{Título del Capítulo Tres}

% Para hacer el índice del Glosario de Términos
\usepackage{makeidx}
\makeindex

%% ----------------------------------------------------------------
\begin{document}
\frontmatter      % Comenzar estilo romano (i, ii, iii, iv...) para el numerado de páginas

% Aquí se importan las primeras páginas, puede encontrarlas en la carpeta Páginas_Princ
% Los ficheros en la subcarpeta Fijo no necesitan edición.
% Si no necesita ninguna de estas secciones, simplemente comente, o elimine la fila input

%% Todas las páginas antes de los capítulos ------------------------------
% Configurar la Página de Título - NO EDITAR ESTO, (si no quiere;)  )
% en vez especificar el nombre, título y más en  "/Configuraciones/Administrativo.tex"
\title   {\myTitle}
\authors {\texorpdfstring
            {\href{\myAuthorEmail}{\myAuthorName}}
            {\myAuthorName}
         }
\addresses  {\groupname\\\deptname\\\univname}  
\date       {\myDate}
\subject    {\mySubject}
\keywords   {\myKeywords}

\maketitle
%% ----------------------------------------------------------------

\setstretch{1.3}  % It is better to have smaller font and larger line spacing than the other way round

% Define the page headers using the FancyHdr package and set up for one-sided printing
\fancyhead{}  % Clears all page headers and footers
\rhead{\thepage}  % Sets the right side header to show the page number
\lhead{}  % Clears the left side page header


%% ----------------------------------------------------------------
% Página de Declaración de Autoría requerida para la Tesis, su institución puede darle un texto diferente que poner acá
\pagestyle{fancy}  % Por último, implementar los encabezados FancyHdr
\clearpage
\Declaration{

\addtocontents{toc}{\vspace{1em}}  % Incluir un espacio en los Contenidos, por estética

Yo, \myAuthorName, declaro que esta tesis titulada `\myTitle' y la investigación presentada en ella son de mi autoría. Confirmo que:

\begin{itemize} 
\item[\tiny{$\blacksquare$}] Este trabajo ha sido realizado completamente para la candidatura de un grado científico en esta Universidad.
 
\item[\tiny{$\blacksquare$}] Donde he consultado el trabajo publicado por otros, siempre se ha referenciado claramente.
 
\item[\tiny{$\blacksquare$}] Donde he citado el trabajo de otros, siempre se ha especificado la fuente. Con la excepción de estas citas, esta tesis es completamente un trabajo realizado por mí.
 
\item[\tiny{$\blacksquare$}] He reconocido todas las fuentes principales de ayuda.
 
\item[\tiny{$\blacksquare$}] Donde la tesis está basada en trabajo realizado por mi persona de conjunto con otros, he dejado claro exactamente qué hicieron otros y qué he contribuido yo misma.

\end{itemize}
 
\vspace{10 mm}
 
Firmado:\\
\rule[1em]{25em}{0.5pt}  % Esto imprime una línea para la firma

Fecha:\\
\rule[1em]{25em}{0.5pt}  % Esto imprime una línea para escribir la fecha
}



% Página para poner una cita especial
\clearpage
\pagestyle{empty}  % Sin encabezado o pie de página para las siguientes páginas

%use 1 o vfill para posicionar la cita donde luzca bien:
\null\vfill\vfill

% Ahora viene la cita, escrita en cursiva:
\begin{flushright}
\textit{
    % Escribir una cita o un pensamiento acá:
    Página intencionalmente en blanco
}
    % Si la cita es tomada de otra persona, sus nombres van aquí:
\end{flushright}


 
\vfill\vfill\vfill\vfill\vfill\null


% Página de Agradecimientos, para agradecer a todos
\clearpage
\setstretch{1.3} % Resetear el espaciado a 1.3 para el cuerpo del texto (si cambia)
\acknowledgements{
	%\addtocontents{toc}{\vspace{1em}} % Añadir un espacio en los contenidos, por estética
	
	% Los agradecimientos y las personas van aquí, no olvidar incluir al supervisor del proyecto, tutores...
	
	A todas las personas que el tesista considere que le ayudaron en su investigación para la culminación de estudios.
}


\clearpage
\lhead{}  % Dejar vacío el encabezado superior izquierdo.
\setstretch{1.3}  % Devolver el espaciado a 1.3
\pagestyle{empty}  % El estilo de página necesita estar vacío para esta página

\dedicatory{
	\addtocontents{toc}{\vspace{1em}} % Añadir un espacio en los contenidos, por estética
	Dedicatoria a las personas que el tesista considere importantes.}



\clearpage 
\addtotoc{Síntesis}  % Adiciona la entrada "Síntesis" a los Contenidos

\thispagestyle{empty}
%\null\vfill
\begin{center}
	\setlength{\parskip}{0pt}
	\bigskip
	{\large{\textit{Síntesis}} \par}
	\bigskip
\end{center}

Aquí se escribe la síntesis de la investigación.

\paragraph{Palabras clave:} clave, palabras


% La Página Resumen (en inglés)
\clearpage 
\addtotoc{Abstract}  % Adiciona la entrada "Abstract" a los Contenidos
\addtocontents{toc}{\vspace{1em}}  % Incluir un espacio en los Contenidos, por estética

\abstract{
	
	The Thesis Abstract is written here (and usually kept to just this page). 
	
	The page is kept centered vertically so can expand into the blank space above the title too\ldots
	
}

\keywords{key, words}
\paragraph{Keywords:}key, words

\clearpage % Comenzar una nueva página
\setstretch{1.3} % Reestablecer el espaciado a 1.3 para el cuerpo del texto (si cambia)
\pagestyle{fancy} % Los estilos de encabezado de página han estado "vacíos" todo este tiempo, 
                  % ahora se usan los encabezados "fancy" como se definió antes
\lhead{\emph{Contenidos}}  % Cambiar el encabezado izquierdo superior a "Contenidos"
\setcounter{tocdepth}{3} % Para que tome hasta el tercer nivel de texto en los Contenidos
\renewcommand{\contentsname}{Contenidos}
\tableofcontents  % Escribir la tabla de contenidos


\setstretch{1.3} % Reset the line-spacing to 1.3 for body text (if changed)
\pagestyle{fancy} % The page style headers have been "empty" all this time, 
                  % now use the "fancy" headers as defined before
\lhead{\emph{\'Indice de Figuras}}  % Cambiar el encabezado izquierdo superior a "Indice de Figuras"
\renewcommand{\listfigurename}{\'Indice de Figuras}
\listoffigures  % Write out the List of Figures


\clearpage  % Start a new page
\setstretch{1.3} % Reset the line-spacing to 1.3 for body text (if changed)
\pagestyle{fancy} % The page style headers have been "empty" all this time, 
                  % now use the "fancy" headers as defined before
\lhead{\emph{\'Indice de Tablas}}  % Cambiar el encabezado izquierdo superior a "Lista de Tablas"
\renewcommand{\listtablename}{\'Indice de Tablas}
\listoftables  % Write out the List of Tables


%% El Cuerpo -------------------------------------------------------
\setstretch{1.3}  % Devolver el espaciado de línea a 1.3
\mainmatter	  % Comenzar el numerado de página normal (1,2,3...)
\pagestyle{fancy}  % Devolver los encabezados de página al estilo "fancy"

\renewcommand{\chaptername}{Capítulo}
% Incluír los capítulos de la tesis, como ficheros separados
% Simplemente quite el símbolo de comentario a las líneas a medida
% que vaya escribiendo los capítulos

\addtocontents{toc}{\vspace{1em}} % Añadir un espacio en los contenidos, por estética

\lhead{\emph{Introducción}}  % Establecer el encabezado superior izquierdo a "Introducción"
% Para que no le ponga delante a la Introducción la palabra Capítulo
\chapter*{Introducción}
\addcontentsline{toc}{chapter}{Introducción}

Lorem ipsum dolor sit amet, consectetur adipiscing elit, sed do eiusmod tempor incididunt ut labore et dolore magna aliqua. Ut enim ad minim veniam, quis nostrud exercitation ullamco laboris nisi ut aliquip ex ea commodo consequat. Duis aute irure dolor in reprehenderit in voluptate velit esse cillum dolore eu fugiat nulla pariatur. Excepteur sint occaecat cupidatat non proident, sunt in culpa qui officia deserunt mollit anim id est laborum \cite{noauthor_algo_2019}.

Ejemplo de una tabla

\begin{table}[ht] 
	\centering 
	\begin{tabular}{ | l | l | l | p{5cm} |}
		\hline
		\textbf{Criterio} & \textbf{Criterio} & \textbf{Criterio} \\ \hline
		Algo & Algo & Algo  \\ \hline
		Algo & Algo & Algo  \\
		\hline
	\end{tabular}
	\caption[Titular de la tabla]{Titular de la tabla\footnote{Ejemplo de nota al pie}}
	\label{table:atable}
\end{table} 

Lorem ipsum dolor sit amet, consectetur adipiscing elit, sed do eiusmod tempor incididunt ut labore et dolore magna aliqua. Ut enim ad minim veniam, quis nostrud exercitation ullamco laboris nisi ut aliquip ex ea commodo consequat. Duis aute irure dolor in reprehenderit in voluptate velit esse cillum dolore eu fugiat nulla pariatur. Excepteur sint occaecat cupidatat non proident, sunt in culpa qui officia deserunt mollit anim id est laborum \ref{fig:afigure}.

Ejemplo de caja de texto:

\begin{center}
	{ \fboxsep 8pt
		\fcolorbox {black}{white}{
			\begin{minipage}[t]{16cm}
				\begin{enumerate}
					\item Lorem ipsum dolor sit amet, consectetur adipiscing elit, sed do eiusmod tempor incididunt ut labore et dolore magna aliqua
				\end{enumerate}
			\end{minipage}
		} }
	\end{center}

\begin{figure}
	\centering
	\includegraphics[width=10cm,height=10cm]{Figuras/ucilogo.png}
	\caption{Título de la Figura}
	\label{fig:afigure}
\end{figure}

Atendiendo a lo expresado anteriormente se plantea el siguiente \textbf{problema científico}: ......

Para dar solución al problema científico se procede a definir una \textbf{metodología} para guiar la investigación, optando por la investigación cuantitativa experimental, con paradigma de investigación positivista. Esto permitirá definir y comprobar la hipótesis de investigación durante el transcurso de la misma, utilizando parámetros medibles y cuantificables. De esta forma, sustentada en la idea de explicar, se buscará encontrar el modelo que evidencie esto en la realidad social.

Para ello se determina como \textbf{objeto de estudio} el ....., especificándose como \textbf{campo de acción} el .....

El \textbf{objetivo general} de la presente investigación será ....., teniendo como \textbf{objetivos específicos}:
\begin{enumerate}
	\item Lorem ipsum dolor sit amet, consectetur adipiscing elit, sed do eiusmod tempor incididunt ut labore et dolore magna aliqua
\end{enumerate}

Según lo expuesto y teniendo en cuenta el marco teórico referencial de la presente investigación, se plantea como \textbf{hipótesis de trabajo} que si ...... Se determina como \textbf{variable independiente} .... y como \textbf{variable dependiente} .....

La presente investigación está sustentada sobre la base de la utilización de diferentes \textbf{métodos científicos}. Los \textbf{métodos teóricos} utilizados fueron:

\begin{itemize}
	\item Método 1: explicación.
\end{itemize}

Los \textbf{métodos empíricos} utilizados fueron:
\begin{itemize}
	\item Método 1: explicación.
\end{itemize}

Los \textbf{aportes teóricos} del trabajo lo constituyen:
\begin{enumerate}
	\item Aporte 1
\end{enumerate}

Los \textbf{aportes prácticos} del trabajo lo constituyen:
\begin{enumerate}
	\item Aporte 1
\end{enumerate}

La investigación consta de un resumen en idiomas español e inglés, contenidos, índice de fi\-guras, índice de tablas, abreviaturas, glosario de términos, introducción, tres capítulos, conclusiones, recomendaciones, referencias bibliográficas, bibliografía consultada y un cuerpo de anexos. Los tres capítulos están organizados de la siguiente forma:

\textbf{CAPÍTULO I}: “\CapUno”: explicación del capítulo.

\textbf{CAPÍTULO II}: “\CapDos”: explicación del capítulo.

\textbf{CAPÍTULO III}: “\CapTres”: explicación del capítulo. % Introducción

% Para todos los capítulos, use la funcionalidad nueva chap{} en vez de chapter{}
% Esto hará que el texto en la izquierda superior de la página sea el mismo que el del capítulo

% Los Nombres de los capítulos se definen en Configuraciones/Administrativo.tex
\chap{\CapUno}

\section{El sistema operativo GNU/Linux y sus distribuciones}
Lorem ipsum dolor sit amet, consectetur adipiscing elit, sed do eiusmod tempor incididunt ut labore et dolore magna aliqua. Ut enim ad minim veniam, quis nostrud exercitation ullamco laboris nisi ut aliquip ex ea commodo consequat. Duis aute irure dolor in reprehenderit in voluptate velit esse cillum dolore eu fugiat nulla pariatur. Excepteur sint occaecat cupidatat non proident, sunt in culpa qui officia deserunt mollit anim id est laborum \cite{noauthor_algo_2019}.

\begin{figure}
	\centering
	\includegraphics[width=10cm,height=10cm]{Figuras/ucilogo.png}
	\caption{Título de la Figura}
\end{figure}

\subsection{Un ejemplo}
de listas no numeradas.

\begin{itemize}
	\item elemento 1.
\end{itemize}

\section{Sección 1}
Lorem ipsum dolor sit amet, consectetur adipiscing elit, sed do eiusmod tempor incididunt ut labore et dolore magna aliqua. Ut enim ad minim veniam, quis nostrud exercitation ullamco laboris nisi ut aliquip ex ea commodo consequat. Duis aute irure dolor in reprehenderit in voluptate velit esse cillum dolore eu fugiat nulla pariatur. Excepteur sint occaecat cupidatat non proident, sunt in culpa qui officia deserunt mollit anim id est laborum \ref{fig:afigure}.

\subsection{Subsección 1}
Lorem ipsum dolor sit amet, consectetur adipiscing elit, sed do eiusmod tempor incididunt ut labore et dolore magna aliqua.
 % 

% \input{Capitulos/Capitulo2} % 

% \input{Capitulos/Capitulo3} % 

%% Conclusiones ---------------------------------------------------
% Para que no le ponga delante a las Conclusiones la palabra Capítulo
\chapter*{Conclusiones}
%Adicionar a la tabla de contenidos la entrada Conclusiones
\addcontentsline{toc}{chapter}{Conclusiones}

\begin{enumerate}
	\item una conclusión.
\end{enumerate}

%% Recomendaciones ---------------------------------------------------
% Para que no le ponga delante a las Recomendaciones la palabra Capítulo
\chapter*{Recomendaciones}

\addcontentsline{toc}{chapter}{Recomendaciones}

\begin{enumerate}
	\item una recomendación.
\end{enumerate}


%% Referencias bibliográficas ---------------------------------------------------
\backmatter % Pista para decirle a LaTeX que los siguientes 'capítulos' son Referencias bibliográficas
\label{Referencias}
\lhead{\emph{Referencias Bibliográficas}}  % Cambiar encabezado izquierdo a "Referencias bibliográficas"
\addtotoc{Referencias Bibliográficas}
\renewcommand\bibname{Referencias Bibliográficas}
% Se pueden usar como estilos: plain, apalike, alpha, abbrv, unsrt
\bibliographystyle{apalike}  % Usar el estilo de BibTeX para el formato de la Bibliografía
% plain, apalike, alpha, abbrv, unsrt.

\begingroup
    \raggedright
    \sloppy
    \bibliography{Referencias}  % La información de las referencias bibliográficas se almacenan en el fichero llamado "Referencias.bib"
\endgroup 

\clearpage

%% -----------------------------------------------------

\clearpage
\pagestyle{fancy} % Los estilos de encabezados de página han estado "vacíos" todo este tiempo, 
% ahora se usan los encabezados "bonitos" como se definió anteriormente
\setstretch{1.5} % Establecer el espaciado a 1.5, 
% esto hace que las siguientes tablas sean más fáciles de leer
\lhead{\emph{Producci\'on Cient\'ifica del Autor}}  % Establecer el encabezado superior izquierdo
\addtotoc{Producci\'on Cient\'ifica}

\chapter*{Producci\'on Cient\'ifica del Autor}

\textbf{\large{Publicaciones en revistas}}
\begin{enumerate}
	\item Publicaciones científicas del autor. Esta sección puede ser obviada. Sólo es requerida en tesis de maestría y doctorado.
\end{enumerate}

\textbf{\large{Publicaciones en memorias de eventos}}
\begin{enumerate}
	\item Publicaciones científicas del autor. Esta sección puede ser obviada. Sólo es requerida en tesis de maestría y doctorado.
\end{enumerate}

\clearpage
\pagestyle{fancy} % Los estilos de encabezados de página han estado "vacíos" todo este tiempo, 
                  % ahora se usan los encabezados "bonitos" como se definió anteriormente
\setstretch{1.5} % Establecer el espaciado a 1.5, 
                 % esto hace que las siguientes tablas sean más fáciles de leer
\lhead{\emph{Glosario de Términos}}  % Establecer el encabezado superior izquierdo a "Glosario de Términos"

% Para generar el glosario de términos, ir a Herramientas > Órdenes > PDFLaTeX, luego Herramientas > Órdenes > MakeIndex > y por último Herramientas > Órdenes > PDFLaTeX
\addtotoc{Glosario de Términos}
\renewcommand{\indexname}{Glosario de Términos}
\printindex

\cleardoublepage

%% Anexos -----------------------------------------------------
\addtocontents{toc}{\vspace{2em}} % Añadir un espacio en los Contenidos, por estética
\lhead{\emph{Anexos}}  % Cambiar el encabezado izquierdo a "Anexos"
\appendix % Pista para decirle a LaTeX que los siguientes 'capítulos' son Anexos

\input{Anexos/AnexoA} % Título del Anexo

%\input{Anexos/AnexoB} % Título del Anexo

%\input{Anexos/AnexoC} % Título del Anexo

\end{document}  % Fin
%% ----------------------------------------------------------------