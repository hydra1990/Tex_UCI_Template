% Para todos los capítulos, use la funcionalidad nueva chap{} en vez de chapter{}
% Esto hará que el texto en la izquierda superior de la página sea el mismo que el del capítulo

% Los Nombres de los capítulos se definen en Configuraciones/Administrativo.tex
\chap{\CapUno}

\section{El sistema operativo GNU/Linux y sus distribuciones}
Lorem ipsum dolor sit amet, consectetur adipiscing elit, sed do eiusmod tempor incididunt ut labore et dolore magna aliqua. Ut enim ad minim veniam, quis nostrud exercitation ullamco laboris nisi ut aliquip ex ea commodo consequat. Duis aute irure dolor in reprehenderit in voluptate velit esse cillum dolore eu fugiat nulla pariatur. Excepteur sint occaecat cupidatat non proident, sunt in culpa qui officia deserunt mollit anim id est laborum \cite{noauthor_algo_2019}.

\begin{figure}
	\centering
	\includegraphics[width=10cm,height=10cm]{Figuras/ucilogo.png}
	\caption{Título de la Figura}
\end{figure}

\subsection{Un ejemplo}
de listas no numeradas.

\begin{itemize}
	\item elemento 1.
\end{itemize}

\section{Sección 1}
Lorem ipsum dolor sit amet, consectetur adipiscing elit, sed do eiusmod tempor incididunt ut labore et dolore magna aliqua. Ut enim ad minim veniam, quis nostrud exercitation ullamco laboris nisi ut aliquip ex ea commodo consequat. Duis aute irure dolor in reprehenderit in voluptate velit esse cillum dolore eu fugiat nulla pariatur. Excepteur sint occaecat cupidatat non proident, sunt in culpa qui officia deserunt mollit anim id est laborum \ref{fig:afigure}.

\subsection{Subsección 1}
Lorem ipsum dolor sit amet, consectetur adipiscing elit, sed do eiusmod tempor incididunt ut labore et dolore magna aliqua.
