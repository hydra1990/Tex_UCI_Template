\addtocontents{toc}{\vspace{1em}} % Añadir un espacio en los contenidos, por estética

\lhead{\emph{Introducción}}  % Establecer el encabezado superior izquierdo a "Introducción"
% Para que no le ponga delante a la Introducción la palabra Capítulo
\chapter*{Introducción}
\addcontentsline{toc}{chapter}{Introducción}

Lorem ipsum dolor sit amet, consectetur adipiscing elit, sed do eiusmod tempor incididunt ut labore et dolore magna aliqua. Ut enim ad minim veniam, quis nostrud exercitation ullamco laboris nisi ut aliquip ex ea commodo consequat. Duis aute irure dolor in reprehenderit in voluptate velit esse cillum dolore eu fugiat nulla pariatur. Excepteur sint occaecat cupidatat non proident, sunt in culpa qui officia deserunt mollit anim id est laborum \cite{noauthor_algo_2019}.

Ejemplo de una tabla

\begin{table}[ht] 
	\centering 
	\begin{tabular}{ | l | l | l | p{5cm} |}
		\hline
		\textbf{Criterio} & \textbf{Criterio} & \textbf{Criterio} \\ \hline
		Algo & Algo & Algo  \\ \hline
		Algo & Algo & Algo  \\
		\hline
	\end{tabular}
	\caption[Titular de la tabla]{Titular de la tabla\footnote{Ejemplo de nota al pie}}
	\label{table:atable}
\end{table} 

Lorem ipsum dolor sit amet, consectetur adipiscing elit, sed do eiusmod tempor incididunt ut labore et dolore magna aliqua. Ut enim ad minim veniam, quis nostrud exercitation ullamco laboris nisi ut aliquip ex ea commodo consequat. Duis aute irure dolor in reprehenderit in voluptate velit esse cillum dolore eu fugiat nulla pariatur. Excepteur sint occaecat cupidatat non proident, sunt in culpa qui officia deserunt mollit anim id est laborum \ref{fig:afigure}.

Ejemplo de caja de texto:

\begin{center}
	{ \fboxsep 8pt
		\fcolorbox {black}{white}{
			\begin{minipage}[t]{16cm}
				\begin{enumerate}
					\item Lorem ipsum dolor sit amet, consectetur adipiscing elit, sed do eiusmod tempor incididunt ut labore et dolore magna aliqua
				\end{enumerate}
			\end{minipage}
		} }
	\end{center}

\begin{figure}
	\centering
	\includegraphics[width=10cm,height=10cm]{Figuras/ucilogo.png}
	\caption{Título de la Figura}
	\label{fig:afigure}
\end{figure}

Atendiendo a lo expresado anteriormente se plantea el siguiente \textbf{problema científico}: ......

Para dar solución al problema científico se procede a definir una \textbf{metodología} para guiar la investigación, optando por la investigación cuantitativa experimental, con paradigma de investigación positivista. Esto permitirá definir y comprobar la hipótesis de investigación durante el transcurso de la misma, utilizando parámetros medibles y cuantificables. De esta forma, sustentada en la idea de explicar, se buscará encontrar el modelo que evidencie esto en la realidad social.

Para ello se determina como \textbf{objeto de estudio} el ....., especificándose como \textbf{campo de acción} el .....

El \textbf{objetivo general} de la presente investigación será ....., teniendo como \textbf{objetivos específicos}:
\begin{enumerate}
	\item Lorem ipsum dolor sit amet, consectetur adipiscing elit, sed do eiusmod tempor incididunt ut labore et dolore magna aliqua
\end{enumerate}

Según lo expuesto y teniendo en cuenta el marco teórico referencial de la presente investigación, se plantea como \textbf{hipótesis de trabajo} que si ...... Se determina como \textbf{variable independiente} .... y como \textbf{variable dependiente} .....

La presente investigación está sustentada sobre la base de la utilización de diferentes \textbf{métodos científicos}. Los \textbf{métodos teóricos} utilizados fueron:

\begin{itemize}
	\item Método 1: explicación.
\end{itemize}

Los \textbf{métodos empíricos} utilizados fueron:
\begin{itemize}
	\item Método 1: explicación.
\end{itemize}

Los \textbf{aportes teóricos} del trabajo lo constituyen:
\begin{enumerate}
	\item Aporte 1
\end{enumerate}

Los \textbf{aportes prácticos} del trabajo lo constituyen:
\begin{enumerate}
	\item Aporte 1
\end{enumerate}

La investigación consta de un resumen en idiomas español e inglés, contenidos, índice de fi\-guras, índice de tablas, abreviaturas, glosario de términos, introducción, tres capítulos, conclusiones, recomendaciones, referencias bibliográficas, bibliografía consultada y un cuerpo de anexos. Los tres capítulos están organizados de la siguiente forma:

\textbf{CAPÍTULO I}: “\CapUno”: explicación del capítulo.

\textbf{CAPÍTULO II}: “\CapDos”: explicación del capítulo.

\textbf{CAPÍTULO III}: “\CapTres”: explicación del capítulo.